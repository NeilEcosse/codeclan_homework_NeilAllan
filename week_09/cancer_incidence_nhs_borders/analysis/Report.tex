% Options for packages loaded elsewhere
\PassOptionsToPackage{unicode}{hyperref}
\PassOptionsToPackage{hyphens}{url}
%
\documentclass[
]{article}
\usepackage{lmodern}
\usepackage{amssymb,amsmath}
\usepackage{ifxetex,ifluatex}
\ifnum 0\ifxetex 1\fi\ifluatex 1\fi=0 % if pdftex
  \usepackage[T1]{fontenc}
  \usepackage[utf8]{inputenc}
  \usepackage{textcomp} % provide euro and other symbols
\else % if luatex or xetex
  \usepackage{unicode-math}
  \defaultfontfeatures{Scale=MatchLowercase}
  \defaultfontfeatures[\rmfamily]{Ligatures=TeX,Scale=1}
\fi
% Use upquote if available, for straight quotes in verbatim environments
\IfFileExists{upquote.sty}{\usepackage{upquote}}{}
\IfFileExists{microtype.sty}{% use microtype if available
  \usepackage[]{microtype}
  \UseMicrotypeSet[protrusion]{basicmath} % disable protrusion for tt fonts
}{}
\makeatletter
\@ifundefined{KOMAClassName}{% if non-KOMA class
  \IfFileExists{parskip.sty}{%
    \usepackage{parskip}
  }{% else
    \setlength{\parindent}{0pt}
    \setlength{\parskip}{6pt plus 2pt minus 1pt}}
}{% if KOMA class
  \KOMAoptions{parskip=half}}
\makeatother
\usepackage{xcolor}
\IfFileExists{xurl.sty}{\usepackage{xurl}}{} % add URL line breaks if available
\IfFileExists{bookmark.sty}{\usepackage{bookmark}}{\usepackage{hyperref}}
\hypersetup{
  hidelinks,
  pdfcreator={LaTeX via pandoc}}
\urlstyle{same} % disable monospaced font for URLs
\usepackage[margin=1in]{geometry}
\usepackage{color}
\usepackage{fancyvrb}
\newcommand{\VerbBar}{|}
\newcommand{\VERB}{\Verb[commandchars=\\\{\}]}
\DefineVerbatimEnvironment{Highlighting}{Verbatim}{commandchars=\\\{\}}
% Add ',fontsize=\small' for more characters per line
\usepackage{framed}
\definecolor{shadecolor}{RGB}{248,248,248}
\newenvironment{Shaded}{\begin{snugshade}}{\end{snugshade}}
\newcommand{\AlertTok}[1]{\textcolor[rgb]{0.94,0.16,0.16}{#1}}
\newcommand{\AnnotationTok}[1]{\textcolor[rgb]{0.56,0.35,0.01}{\textbf{\textit{#1}}}}
\newcommand{\AttributeTok}[1]{\textcolor[rgb]{0.77,0.63,0.00}{#1}}
\newcommand{\BaseNTok}[1]{\textcolor[rgb]{0.00,0.00,0.81}{#1}}
\newcommand{\BuiltInTok}[1]{#1}
\newcommand{\CharTok}[1]{\textcolor[rgb]{0.31,0.60,0.02}{#1}}
\newcommand{\CommentTok}[1]{\textcolor[rgb]{0.56,0.35,0.01}{\textit{#1}}}
\newcommand{\CommentVarTok}[1]{\textcolor[rgb]{0.56,0.35,0.01}{\textbf{\textit{#1}}}}
\newcommand{\ConstantTok}[1]{\textcolor[rgb]{0.00,0.00,0.00}{#1}}
\newcommand{\ControlFlowTok}[1]{\textcolor[rgb]{0.13,0.29,0.53}{\textbf{#1}}}
\newcommand{\DataTypeTok}[1]{\textcolor[rgb]{0.13,0.29,0.53}{#1}}
\newcommand{\DecValTok}[1]{\textcolor[rgb]{0.00,0.00,0.81}{#1}}
\newcommand{\DocumentationTok}[1]{\textcolor[rgb]{0.56,0.35,0.01}{\textbf{\textit{#1}}}}
\newcommand{\ErrorTok}[1]{\textcolor[rgb]{0.64,0.00,0.00}{\textbf{#1}}}
\newcommand{\ExtensionTok}[1]{#1}
\newcommand{\FloatTok}[1]{\textcolor[rgb]{0.00,0.00,0.81}{#1}}
\newcommand{\FunctionTok}[1]{\textcolor[rgb]{0.00,0.00,0.00}{#1}}
\newcommand{\ImportTok}[1]{#1}
\newcommand{\InformationTok}[1]{\textcolor[rgb]{0.56,0.35,0.01}{\textbf{\textit{#1}}}}
\newcommand{\KeywordTok}[1]{\textcolor[rgb]{0.13,0.29,0.53}{\textbf{#1}}}
\newcommand{\NormalTok}[1]{#1}
\newcommand{\OperatorTok}[1]{\textcolor[rgb]{0.81,0.36,0.00}{\textbf{#1}}}
\newcommand{\OtherTok}[1]{\textcolor[rgb]{0.56,0.35,0.01}{#1}}
\newcommand{\PreprocessorTok}[1]{\textcolor[rgb]{0.56,0.35,0.01}{\textit{#1}}}
\newcommand{\RegionMarkerTok}[1]{#1}
\newcommand{\SpecialCharTok}[1]{\textcolor[rgb]{0.00,0.00,0.00}{#1}}
\newcommand{\SpecialStringTok}[1]{\textcolor[rgb]{0.31,0.60,0.02}{#1}}
\newcommand{\StringTok}[1]{\textcolor[rgb]{0.31,0.60,0.02}{#1}}
\newcommand{\VariableTok}[1]{\textcolor[rgb]{0.00,0.00,0.00}{#1}}
\newcommand{\VerbatimStringTok}[1]{\textcolor[rgb]{0.31,0.60,0.02}{#1}}
\newcommand{\WarningTok}[1]{\textcolor[rgb]{0.56,0.35,0.01}{\textbf{\textit{#1}}}}
\usepackage{graphicx,grffile}
\makeatletter
\def\maxwidth{\ifdim\Gin@nat@width>\linewidth\linewidth\else\Gin@nat@width\fi}
\def\maxheight{\ifdim\Gin@nat@height>\textheight\textheight\else\Gin@nat@height\fi}
\makeatother
% Scale images if necessary, so that they will not overflow the page
% margins by default, and it is still possible to overwrite the defaults
% using explicit options in \includegraphics[width, height, ...]{}
\setkeys{Gin}{width=\maxwidth,height=\maxheight,keepaspectratio}
% Set default figure placement to htbp
\makeatletter
\def\fps@figure{htbp}
\makeatother
\setlength{\emergencystretch}{3em} % prevent overfull lines
\providecommand{\tightlist}{%
  \setlength{\itemsep}{0pt}\setlength{\parskip}{0pt}}
\setcounter{secnumdepth}{-\maxdimen} % remove section numbering

\author{}
\date{\vspace{-2.5em}}

\begin{document}

\begin{Shaded}
\begin{Highlighting}[]
\KeywordTok{library}\NormalTok{(tidyverse)}
\KeywordTok{library}\NormalTok{(janitor)}
\KeywordTok{library}\NormalTok{(here)}
\end{Highlighting}
\end{Shaded}

\begin{Shaded}
\begin{Highlighting}[]
\NormalTok{cancer_incidence_data <-}\StringTok{ }
\StringTok{ }\KeywordTok{read_csv}\NormalTok{(}\KeywordTok{here}\NormalTok{(}\StringTok{"data_clean/cancer_incidence_data.csv"}\NormalTok{))}

\NormalTok{population_estimates <-}\StringTok{ }
\StringTok{ }\KeywordTok{read_csv}\NormalTok{(}\KeywordTok{here}\NormalTok{(}\StringTok{"data_clean/population_estimates.csv"}\NormalTok{))}

\NormalTok{data_dictionary <-}\StringTok{ }\KeywordTok{read_csv}\NormalTok{(}\KeywordTok{here}\NormalTok{(}\StringTok{"data_raw/data_dictionary.csv"}\NormalTok{)) }
\end{Highlighting}
\end{Shaded}

\hypertarget{incidence-of-cancer-in-the-nhs-borders-area}{%
\section{\texorpdfstring{{Incidence of cancer in the NHS Borders
area}}{Incidence of cancer in the NHS Borders area}}\label{incidence-of-cancer-in-the-nhs-borders-area}}

\hypertarget{overview}{%
\subsection{\texorpdfstring{{Overview}}{Overview}}\label{overview}}

This report uses publicly available data to give a high-level insight
into the incidence of cancer in the NHS Borders area.

It looks at the following questions:

\begin{itemize}
\item
  How has the population changed over time?
\item
  How has the number of cancer cases changed over time?
\item
  What types of cancer are most prevalent in the region?
\item
  What is the impact of the age of our residents?
\end{itemize}

The sources of figures used in the report are listed in the
\emph{references} section

\hypertarget{how-has-the-population-changed-over-time}{%
\subsection{\texorpdfstring{{How has the population changed over
time?}}{How has the population changed over time?}}\label{how-has-the-population-changed-over-time}}

The graph below for the period 1994-2018 shows how the population has
increased - note that there was a particularly steep rise between 1999
and 2008.

\begin{Shaded}
\begin{Highlighting}[]
\NormalTok{population_estimates }\OperatorTok\StringTok{ }
\StringTok{  }\KeywordTok{filter}\NormalTok{(area_name }\OperatorTok{==}\StringTok{ "Borders"}\NormalTok{) }\OperatorTok\StringTok{ }
\StringTok{  }\KeywordTok{filter}\NormalTok{(gender }\OperatorTok{==}\StringTok{ "All"}\NormalTok{) }\OperatorTok
\StringTok{  }\KeywordTok{filter}\NormalTok{(year }\OperatorTok{>=}\StringTok{ }\DecValTok{1994} \OperatorTok{&}\StringTok{ }\NormalTok{year }\OperatorTok{<=}\StringTok{ }\DecValTok{2018}\NormalTok{) }\OperatorTok\StringTok{ }
\StringTok{  }\KeywordTok{group_by}\NormalTok{(area_name, year) }\OperatorTok\StringTok{ }
\StringTok{  }\KeywordTok{summarise}\NormalTok{(}\DataTypeTok{population_estimate =} \KeywordTok{sum}\NormalTok{(population_estimate)) }\OperatorTok\StringTok{ }
\StringTok{  }\KeywordTok{ggplot}\NormalTok{() }\OperatorTok{+}
\StringTok{  }\KeywordTok{aes}\NormalTok{(}\DataTypeTok{x =}\NormalTok{ year, }\DataTypeTok{y =}\NormalTok{ population_estimate) }\OperatorTok{+}
\StringTok{  }\KeywordTok{geom_line}\NormalTok{(}\DataTypeTok{colour =} \StringTok{"#005EB8"}\NormalTok{, }\DataTypeTok{size  =} \DecValTok{2}\NormalTok{) }\OperatorTok{+}
\StringTok{  }\KeywordTok{theme_minimal}\NormalTok{() }\OperatorTok{+}
\StringTok{  }\KeywordTok{scale_x_continuous}\NormalTok{(}\DataTypeTok{breaks =} \DecValTok{1994}\OperatorTok{:}\DecValTok{2018}\NormalTok{) }\OperatorTok{+}
\StringTok{  }\KeywordTok{scale_y_continuous}\NormalTok{(}\DataTypeTok{labels =}\NormalTok{ scales}\OperatorTok{::}\NormalTok{comma) }\OperatorTok{+}
\StringTok{  }\KeywordTok{theme}\NormalTok{(}\DataTypeTok{axis.text.x =} \KeywordTok{element_text}\NormalTok{(}\DataTypeTok{angle=}\DecValTok{45}\NormalTok{,}\DataTypeTok{hjust=}\DecValTok{1}\NormalTok{)) }\OperatorTok{+}
\StringTok{    }\KeywordTok{labs}\NormalTok{(}
    \DataTypeTok{title =} \StringTok{"Mid-year population estimate over time"}\NormalTok{,}
    \DataTypeTok{subtitle =} \StringTok{"NHS Borders 1994-2018 }\CharTok{\textbackslash{}n}\StringTok{"}\NormalTok{,}
    \DataTypeTok{x =} \StringTok{"}\CharTok{\textbackslash{}n}\StringTok{ Year"}\NormalTok{,}
    \DataTypeTok{y =} \StringTok{"Population estimate }\CharTok{\textbackslash{}n}\StringTok{"}
\NormalTok{  )}
\end{Highlighting}
\end{Shaded}

\includegraphics{Report_files/figure-latex/unnamed-chunk-3-1.pdf}

\hypertarget{how-has-the-number-of-cancer-cases-changed-over-time}{%
\subsection{\texorpdfstring{{How has the number of cancer cases changed
over
time?}}{How has the number of cancer cases changed over time?}}\label{how-has-the-number-of-cancer-cases-changed-over-time}}

This graph shows how cancer cases have risen in the same period:

\begin{Shaded}
\begin{Highlighting}[]
\NormalTok{cancer_incidence_data }\OperatorTok\StringTok{ }
\StringTok{  }\KeywordTok{filter}\NormalTok{(hb_name }\OperatorTok{==}\StringTok{ "NHS Borders"}\NormalTok{) }\OperatorTok\StringTok{ }
\StringTok{  }\KeywordTok{filter}\NormalTok{(cancer_site }\OperatorTok{==}\StringTok{ "All cancer types"}\NormalTok{) }\OperatorTok
\StringTok{  }\KeywordTok{filter}\NormalTok{(sex }\OperatorTok{==}\StringTok{ "All"}\NormalTok{) }\OperatorTok\StringTok{ }
\StringTok{  }\KeywordTok{group_by}\NormalTok{(hb_name, year) }\OperatorTok\StringTok{ }
\StringTok{  }\KeywordTok{summarise}\NormalTok{(}\DataTypeTok{total_incidences_all_ages =} \KeywordTok{sum}\NormalTok{(incidences_all_ages)) }\OperatorTok\StringTok{ }
\StringTok{  }\KeywordTok{ggplot}\NormalTok{() }\OperatorTok{+}
\StringTok{  }\KeywordTok{aes}\NormalTok{(}\DataTypeTok{x =}\NormalTok{ year, }\DataTypeTok{y =}\NormalTok{ total_incidences_all_ages) }\OperatorTok{+}
\StringTok{  }\CommentTok{# NHS colours: https://www.england.nhs.uk/nhsidentity/identity-guidelines/colours/}
\StringTok{  }\KeywordTok{geom_line}\NormalTok{(}\DataTypeTok{colour =} \StringTok{"#005EB8"}\NormalTok{, }\DataTypeTok{size  =} \DecValTok{2}\NormalTok{) }\OperatorTok{+}
\StringTok{  }\KeywordTok{theme_minimal}\NormalTok{() }\OperatorTok{+}
\StringTok{  }\KeywordTok{scale_x_continuous}\NormalTok{(}\DataTypeTok{breaks =}\NormalTok{ (}\KeywordTok{min}\NormalTok{(cancer_incidence_data}\OperatorTok{$}\NormalTok{year)}\OperatorTok{:}\KeywordTok{max}\NormalTok{(cancer_incidence_data}\OperatorTok{$}\NormalTok{year))) }\OperatorTok{+}
\StringTok{  }\KeywordTok{scale_y_continuous}\NormalTok{(}\DataTypeTok{labels =}\NormalTok{ scales}\OperatorTok{::}\NormalTok{comma) }\OperatorTok{+}
\StringTok{  }\KeywordTok{theme}\NormalTok{(}\DataTypeTok{axis.text.x =} \KeywordTok{element_text}\NormalTok{(}\DataTypeTok{angle=}\DecValTok{45}\NormalTok{,}\DataTypeTok{hjust=}\DecValTok{1}\NormalTok{)) }\OperatorTok{+}
\StringTok{  }\KeywordTok{labs}\NormalTok{(}
    \DataTypeTok{title =} \StringTok{"Incidence of all cancer types over time"}\NormalTok{,}
    \DataTypeTok{subtitle =} \StringTok{"NHS Borders 1994-2018 }\CharTok{\textbackslash{}n}\StringTok{"}\NormalTok{,}
    \DataTypeTok{x =} \StringTok{"}\CharTok{\textbackslash{}n}\StringTok{ Year"}\NormalTok{,}
    \DataTypeTok{y =} \StringTok{"Number of cases }\CharTok{\textbackslash{}n}\StringTok{"}
\NormalTok{  )}
\end{Highlighting}
\end{Shaded}

\includegraphics{Report_files/figure-latex/unnamed-chunk-4-1.pdf}

\hypertarget{what-types-of-cancer-are-most-prevalent-in-the-region}{%
\subsection{\texorpdfstring{{What types of cancer are most prevalent in
the
region?}}{What types of cancer are most prevalent in the region?}}\label{what-types-of-cancer-are-most-prevalent-in-the-region}}

The graph below gives a summary of the most common types of cancer: for
the period 1994-2018, it shows totals for the cancer sites where at
least 500 cases were recorded (13 of the 41 cancer types recorded).

\begin{Shaded}
\begin{Highlighting}[]
\NormalTok{cancer_incidence_data }\OperatorTok\StringTok{ }
\StringTok{  }\KeywordTok{filter}\NormalTok{(hb_name }\OperatorTok{==}\StringTok{ "NHS Borders"}\NormalTok{) }\OperatorTok\StringTok{ }
\StringTok{  }\KeywordTok{filter}\NormalTok{(cancer_site }\OperatorTok{!=}\StringTok{ "All cancer types"}\NormalTok{) }\OperatorTok\StringTok{ }
\StringTok{  }\KeywordTok{filter}\NormalTok{(sex }\OperatorTok{==}\StringTok{ "All"}\NormalTok{) }\OperatorTok
\StringTok{  }\KeywordTok{group_by}\NormalTok{(hb_name, cancer_site) }\OperatorTok\StringTok{ }
\StringTok{  }\KeywordTok{summarise}\NormalTok{(}\DataTypeTok{total_incidences_all_ages =} \KeywordTok{sum}\NormalTok{(incidences_all_ages)) }\OperatorTok\StringTok{ }
\StringTok{  }\KeywordTok{filter}\NormalTok{(total_incidences_all_ages }\OperatorTok{>=}\DecValTok{500}\NormalTok{) }\OperatorTok\StringTok{ }
\StringTok{  }\KeywordTok{ggplot}\NormalTok{() }\OperatorTok{+}
\StringTok{  }\KeywordTok{aes}\NormalTok{(}\DataTypeTok{x =} \KeywordTok{reorder}\NormalTok{(cancer_site, }\OperatorTok{-}\NormalTok{total_incidences_all_ages), }\DataTypeTok{y =}\NormalTok{ total_incidences_all_ages) }\OperatorTok{+}
\StringTok{  }\CommentTok{# NHS colours: https://www.england.nhs.uk/nhsidentity/identity-guidelines/colours/}
\StringTok{  }\KeywordTok{geom_col}\NormalTok{(}\DataTypeTok{fill =} \StringTok{"#005EB8"}\NormalTok{) }\OperatorTok{+}
\StringTok{  }\KeywordTok{theme_minimal}\NormalTok{() }\OperatorTok{+}
\StringTok{  }\KeywordTok{scale_y_continuous}\NormalTok{(}\DataTypeTok{labels =}\NormalTok{ scales}\OperatorTok{::}\NormalTok{comma) }\OperatorTok{+}
\StringTok{  }\KeywordTok{theme}\NormalTok{(}\DataTypeTok{axis.text.x =} \KeywordTok{element_text}\NormalTok{(}\DataTypeTok{angle=}\DecValTok{45}\NormalTok{,}\DataTypeTok{hjust=}\DecValTok{1}\NormalTok{)) }\OperatorTok{+}
\StringTok{    }\KeywordTok{labs}\NormalTok{(}
    \DataTypeTok{title =} \StringTok{"Number of cases by cancer site"}\NormalTok{,}
    \DataTypeTok{subtitle =} \StringTok{"NHS Borders 1994-2018 }\CharTok{\textbackslash{}n}\StringTok{"}\NormalTok{,}
    \DataTypeTok{x =} \StringTok{"}\CharTok{\textbackslash{}n}\StringTok{ Cancer site"}\NormalTok{,}
    \DataTypeTok{y =} \StringTok{"Number of cases }\CharTok{\textbackslash{}n}\StringTok{"}\NormalTok{,}
    \DataTypeTok{colour =} \StringTok{""}
\NormalTok{  )}
\end{Highlighting}
\end{Shaded}

\includegraphics{Report_files/figure-latex/unnamed-chunk-5-1.pdf}

\hypertarget{what-is-the-impact-of-the-age-of-our-residents}{%
\subsection{\texorpdfstring{{What is the impact of the age of our
residents?}}{What is the impact of the age of our residents?}}\label{what-is-the-impact-of-the-age-of-our-residents}}

The relative proportion of older people in The Borders is projected to
rise, which will have an impact on the types and capacity of services
which the NHS will need to offer, including cancer services.

To illustrate this changing demographic, the graph of cancer cases below
shows:

\begin{itemize}
\item
  {\textbf{The crude rate:}} This is calculated as the number of cases
  per 100,000 people.
\item
  {\textbf{The European age-standardised rate:}} This is the rate that
  would have been found if the population of The Borders had the same
  age-composition (proportion of total population in each five year age
  class) as a hypothetical European population, known as the \emph{2013
  European Standard Population}.
\end{itemize}

\textbf{The age-standardised rate should not be affected by any changes
in the distribution of the population by age.}

What you can see in this graph is that from 2009 onward, the crude rate
exceeds the age-standardised rate - this higher number of cancer cases
gives an indication that the actual population of The Borders is older
than the standardised version.

There are now proportionally more people in older age groups, who are
more likely to develop cancer; estimates published by National Records
of Scotland \emph{(see reference section below)} anticipate that this
trend will continue, so cancer services in The Borders will need to
increase provision to treat and support these patients.

\begin{Shaded}
\begin{Highlighting}[]
\CommentTok{# Crude rate}
\NormalTok{crude_rate <-}\StringTok{ }\NormalTok{cancer_incidence_data }\OperatorTok\StringTok{ }
\StringTok{  }\KeywordTok{filter}\NormalTok{(hb_name }\OperatorTok{==}\StringTok{ "NHS Borders"}\NormalTok{) }\OperatorTok\StringTok{ }
\StringTok{  }\KeywordTok{filter}\NormalTok{(cancer_site }\OperatorTok{==}\StringTok{ "All cancer types"}\NormalTok{) }\OperatorTok
\StringTok{  }\KeywordTok{filter}\NormalTok{(sex }\OperatorTok{==}\StringTok{ "All"}\NormalTok{) }\OperatorTok\StringTok{ }
\StringTok{  }\KeywordTok{mutate}\NormalTok{(}\DataTypeTok{measure =} \StringTok{"Crude rate"}\NormalTok{,}
         \DataTypeTok{rate =}\NormalTok{ crude_rate) }\OperatorTok\StringTok{ }
\StringTok{  }\KeywordTok{select}\NormalTok{(hb_name, year, measure, rate) }
\CommentTok{# European age-standardised rate}
\NormalTok{european_standardised_rate <-}\StringTok{ }\NormalTok{cancer_incidence_data }\OperatorTok\StringTok{ }
\StringTok{  }\KeywordTok{filter}\NormalTok{(hb_name }\OperatorTok{==}\StringTok{ "NHS Borders"}\NormalTok{) }\OperatorTok\StringTok{ }
\StringTok{  }\KeywordTok{filter}\NormalTok{(cancer_site }\OperatorTok{==}\StringTok{ "All cancer types"}\NormalTok{) }\OperatorTok
\StringTok{  }\KeywordTok{filter}\NormalTok{(sex }\OperatorTok{==}\StringTok{ "All"}\NormalTok{) }\OperatorTok\StringTok{ }
\StringTok{  }\KeywordTok{mutate}\NormalTok{(}\DataTypeTok{measure =} \StringTok{"European age-standardised rate"}\NormalTok{,}
         \DataTypeTok{rate =}\NormalTok{ easr) }\OperatorTok\StringTok{ }
\StringTok{  }\KeywordTok{select}\NormalTok{(hb_name, year, measure, rate)  }

\CommentTok{# put these two datasets together}
\NormalTok{compare_rates <-}\StringTok{ }
\KeywordTok{bind_rows}\NormalTok{(crude_rate, european_standardised_rate)}

\CommentTok{# drop intermediate tables}
\KeywordTok{rm}\NormalTok{(crude_rate, european_standardised_rate)}

\NormalTok{ compare_rates }\OperatorTok\StringTok{  }
\StringTok{  }\KeywordTok{ggplot}\NormalTok{() }\OperatorTok{+}
\StringTok{  }\KeywordTok{aes}\NormalTok{(}\DataTypeTok{x =}\NormalTok{ year, }\DataTypeTok{y =}\NormalTok{ rate, }\DataTypeTok{group  =}\NormalTok{ measure, }\DataTypeTok{colour =}\NormalTok{ measure  ) }\OperatorTok{+}
\StringTok{  }\KeywordTok{geom_line}\NormalTok{(}\DataTypeTok{size  =} \DecValTok{2}\NormalTok{) }\OperatorTok{+}
\StringTok{  }\KeywordTok{theme_minimal}\NormalTok{() }\OperatorTok{+}
\StringTok{  }\KeywordTok{scale_x_continuous}\NormalTok{(}\DataTypeTok{breaks =}\NormalTok{ (}\KeywordTok{min}\NormalTok{(cancer_incidence_data}\OperatorTok{$}\NormalTok{year)}\OperatorTok{:}\KeywordTok{max}\NormalTok{(cancer_incidence_data}\OperatorTok{$}\NormalTok{year))) }\OperatorTok{+}
\StringTok{  }\KeywordTok{scale_y_continuous}\NormalTok{(}\DataTypeTok{labels =}\NormalTok{ scales}\OperatorTok{::}\NormalTok{comma) }\OperatorTok{+}
\StringTok{  }\KeywordTok{theme}\NormalTok{(}\DataTypeTok{axis.text.x =} \KeywordTok{element_text}\NormalTok{(}\DataTypeTok{angle=}\DecValTok{45}\NormalTok{,}\DataTypeTok{hjust=}\DecValTok{1}\NormalTok{)) }\OperatorTok{+}
\StringTok{   }\KeywordTok{theme}\NormalTok{(}\DataTypeTok{legend.position=}\StringTok{"bottom"}\NormalTok{) }\OperatorTok{+}
\StringTok{     }\KeywordTok{scale_colour_manual}\NormalTok{(}
    \DataTypeTok{values =} \KeywordTok{c}\NormalTok{(}
      \StringTok{"Crude rate"}\NormalTok{ =}\StringTok{ "#00A9CE"}\NormalTok{,}
      \StringTok{"European age-standardised rate"}\NormalTok{ =}\StringTok{ "#003087"}
\NormalTok{     )}
\NormalTok{    ) }\OperatorTok{+}
\StringTok{  }\KeywordTok{labs}\NormalTok{(}
    \DataTypeTok{title =} \StringTok{"Crude rate vs European age-standardised rate"}\NormalTok{,}
    \DataTypeTok{subtitle =} \StringTok{"NHS Borders 1994-2018 }\CharTok{\textbackslash{}n}\StringTok{"}\NormalTok{,}
    \DataTypeTok{x =} \StringTok{"}\CharTok{\textbackslash{}n}\StringTok{ Year"}\NormalTok{,}
    \DataTypeTok{y =} \StringTok{"Number of cases all cancers per 100,000 }\CharTok{\textbackslash{}n}\StringTok{"}\NormalTok{,}
    \DataTypeTok{colour =} \StringTok{""}
\NormalTok{  )}
\end{Highlighting}
\end{Shaded}

\includegraphics{Report_files/figure-latex/unnamed-chunk-6-1.pdf}

\hypertarget{summary-of-key-points}{%
\subsection{\texorpdfstring{{Summary of key
points}}{Summary of key points}}\label{summary-of-key-points}}

For the NHS Borders region during the period 1994-2018:

\begin{itemize}
\item
  The population rose from 105,450 in 1994 to 115,270 in 2018
\item
  The number of cancer cases recorded has also tended to increase over
  time - the lowest annual total was 518 in 1994, and the highest was
  912 in 2017
\item
  The five cancer sites with the most cases were:

  \begin{itemize}
  \item
    Non-melanoma skin cancer
  \item
    Basal cell carcinoma of the skin
  \item
    Breast
  \item
    Trachea, bronchus and lung
  \item
    Colorectal cancer
  \end{itemize}
\item
  The proportion of older people in the population rose, and this trend
  is projected to continue
\end{itemize}

\hypertarget{references}{%
\subsubsection{\texorpdfstring{{References}}{References}}\label{references}}

\textbf{Data on cancer instance over time:}
\url{https://www.opendata.nhs.scot/dataset/annual-cancer-incidence/resource/3aef16b7-8af6-4ce0-a90b-8a29d6870014}

\textbf{Population estimate data: }
\url{https://www.nrscotland.gov.uk/statistics-and-data/statistics/statistics-by-theme/population/population-estimates/mid-year-population-estimates/population-estimates-time-series-data}

\textbf{Scottish Borders population trends:}
\url{https://www.nrscotland.gov.uk/files/statistics/council-area-data-sheets/scottish-borders-council-profile.html}

\textbf{Definitions of crude and age-standardised rates:}
\url{https://ecis.jrc.ec.europa.eu/info/glossary.html}

\textbf{Example of using age-standardised rates:}
\url{https://www.nrscotland.gov.uk/files//statistics/age-standardised-death-rates-esp/2017/age-standardised-17-methodology.pdf}

\textbf{Health Board names reference data:}
\url{https://www.opendata.nhs.scot/dataset/9f942fdb-e59e-44f5-b534-d6e17229cc7b/resource/652ff726-e676-4a20-abda-435b98dd7bdc/download/geography_codes_and_labels_hb2014_01042019.csv}

\end{document}
